\documentclass{llncs}
\usepackage{amsmath,amssymb}
\usepackage{graphicx}        % standard LaTeX graphics tool
\usepackage{tikz}        % standard LaTeX graphics tool
\usepackage{subfigure}                                 % authors: subfigures
\usepackage[ruled,vlined,linesnumbered]{algorithm2e}   % authors: last version of algorithm display
\usepackage{todonotes}


%\title{A Subquadratic and Separable Algorithm for Chamfer Norm
  %Distance Transformation \thanks{This work has been mainly funded
  %  by  ANR-11-BS02-009 and  ANR-11-IDEX-0007-02 PALSE/2013/21 research grants.}}



\title{Generic Separable Algorithms for Distance Transformation: A
  Subquadratic Approach for Path Based Norms\thanks{This work has been
    mainly funded by ANR-11-BS02-009 and ANR-11-IDEX-0007-02
    PALSE/2013/21 research grants.}}

\author{David Coeurjolly\inst{1}}

 \institute{ CNRS,  LIRIS, UMR5205, F-69621, France\\
   \email{david.coeurjolly@liris.cnrs.fr}
}
\graphicspath{{./Figs/}}
%fonts bonanza
\usepackage{amsmath,amssymb,amsfonts}
\usepackage{pifont}% http://ctan.org/pkg/pifont
\newcommand{\CheckMark}{\ding{51}}%
\newcommand{\CrossMark}{\ding{55}}%
% Zapf font
\usepackage[mathscr]{euscript}
\DeclareFontFamily{OT1}{pzc}{}
\DeclareFontShape{OT1}{pzc}{m}{it}%
              {<-> s * [1.2] pzcmi7t}{}
\DeclareMathAlphabet{\mathpzc}{OT1}{pzc}{m}{it}
% rescaling cal to be a touch smaller
\DeclareFontFamily{OMS}{fcmsy}{\skewchar\font48 }
\DeclareFontShape{OMS}{fcmsy}{m}{n}{%
         <-5.5> [.96] cmsy5     <5.5-6.5> [.96] cmsy6
      <6.5-7.5> [.96] cmsy7     <7.5-8.5> [.96] cmsy8
      <8.5-9.5> [.96] cmsy9     <9.5->  [.96] cmsy10
      }{}
\DeclareFontShape{OMS}{fcmsy}{b}{n}{%
       <-6> [.96] cmbsy5
      <6-8> [.96] cmbsy7
      <8->  [.96] cmbsy10
      }{}
\DeclareMathAlphabet{\mathcal}{OMS}{fcmsy}{m}{n}
\usepackage{bbm}

\usepackage{mathtools}% http://ctan.org/pkg/mathtools
\usepackage{calc}% http://ctan.org/pkg/calc

\newcommand*{\mytilde}[2][0pt]{%
  \setbox0=\hbox{$#2$}%
  \tilde{\mathrlap{\phantom{\rule{\wd0}{\ht0+{#1}}}}\smash{#2}}%
}
\newcommand*{\mywidetilde}[2][0pt]{%
  \setbox0=\hbox{$#2$}%
  \widetilde{\mathrlap{\phantom{\rule{\wd0}{\ht0+{#1}}}}\smash{#2}}%
}

%%Space, Lattices
\newcommand{\Z}{{\mathbb{Z}}}
\newcommand{\R}{{\mathbb{R}}}

%%Formulas
\newcommand{\EqDef}{\!\ensuremath{\mathrel{\mathop:}=}\!}
%\newcommand{\EqDef}{\smash{\ensuremath{\stackrel{\text{def}}{=}}}}

%% Misc.
\newcommand{\txtblue}[1]{\textcolor{blue}{ #1}}
\newcommand{\txtgreen}[1]{\textcolor{green}{ #1}}
\newcommand{\txtred}[1]{\textcolor{red}{ #1}}

\begin{document}
\maketitle


\begin{abstract}
In many applications, separable algorithms have demonstrated their
efficiency to perform high performance volumetric computations, such
as distance transformation or medial axis extraction. In the
literature, several authors have discussed about the conditions on the
metric to be considered in a separable approach.  In this article, we
present generic separable algorithms to efficiently compute Voronoi
maps and distance transformations for a large class of
metrics. Focusing to path based norms (chamfer masks, neighborhood
sequences, ...), we detail a subquadratic algorithm to compute such
volumetric transformations. More precisely, we describe a
$O(N\log^2(m))$ algorithm in dimension 2 for shapes with $N$ grid
points and for example a chamfer mask of size $m$.

\keywords{Digital Geometry, Distance Transformation, Chamfer Norms}
\end{abstract}

\section{Introduction}
\label{sec:introduction}

Since early works on digital geometry, distance transformation plays
an important role in many applications
\cite{Rosenfeld1966,Rosenfeld1968}. Given a finite input shape
$X\subset \Z^n$, the distance transformation labels each point in $X$
with the distance to the closest point in $\Z^n \setminus X$. Since
such characterization is parametrized by a distance function, many
works have been done to address this distance transformation problem
with a trade-off between algorithmic performances and the
\emph{accuracy} of the digital distance function with respect to the
Euclidean one.  Hence, authors have considered distances based on
chamfer masks
\cite{Rosenfeld1968,borgefors,Thiel_IWCIA7,fouard:ivc:2005} or
sequences of chamfer masks
\cite{ROSEN_66,mukherjee,Nagy05,Strand2008,DBLP:conf/dgci/NormandSE13};
the vector displacement based Euclidean distance
\cite{danielson,ragnemalm,MULL_92,Cuisenaire1999_268}; the Voronoi
diagram based Euclidean distance
\cite{BreuEtAl95,GotLin95,Guan,maurer_pami} or the square of the
Euclidean distance \cite{SaitTori:94,Hirata,roerdnik}.

\paragraph{Contributions} In this article,

\section{Preliminaries}
\label{sec:preliminaries}

Let us first present some definitions.
\begin{definition}[Metrics and Metric Space]
  \label{def:distance}
  A metric is a map $d$ from $E$ to a sub-group $F$ of $\R$ such that
  $\forall a,b,c\in E$,
  \begin{align}
    (\text{non-negative})\quad & d(a,b)\geq 0\\
    (\text{identity of indiscernibles}) \quad&  d(a,b)= 0
    \Leftrightarrow a=b\\
    (\text{symmetry})\quad &  d(a,b)=d(b,a)\\
    (\text{triangular inequality})\quad &   d(a,c) \leq d(a,b) + d(b,c)
  \end{align}
$(E, F, d)$ is called a metric space.
\end{definition}
If property $(4)$ is omitted, we are defining a distance function
instead of a metric. Note that in the following, we will only consider
metric spaces with properties $(1-4)$. Furthermore, we may have to
consider  metrics induced by a vector space norm:
\begin{definition}[Norm and metric induced by a norm]
  \label{def:distance}
  Given a vector space EV, a norm is a map $g$ from  $EV$ to a sub-group
  $F$ of $\R$ such that $\forall \vec{x},\vec{y}\in EV$,
  \begin{align}
    (\text{non-negative})\quad & g(\vec{x})\geq 0\\
    (\text{identity of indiscernibles}) \quad&  g(\vec{x})= 0
    \Leftrightarrow \vec{x}=\vec{0}\\
    (\text{triangular inequality})\quad &   g(\vec{x}+\vec{y}) \leq
    g(\vec{x}) + g(\vec{y})\\
    (\text{homogeneity})\quad &  \forall \lambda\in\R, \quad
    g(\lambda\cdot\vec{x}) = |\lambda|\cdot g(\lambda\cdot\vec{x})
  \end{align}
$d(a,b) \EqDef g(b-a)$ is the metric induced by the
  norm $g$.
\end{definition}
Note that the above definition can be extended from vector spaces to
\emph{modules} on a commutative ring ($\Z^n$ being a module on $\Z$
but not a vector space) \cite{Thiel_hdr}.

From the literature, path-based approaches (Chamfer masks, (weighted)
neighborhood sequences...)  aim at defining metrics induced by norms
in metric spaces $(\Z^n, \Z , d)$.  Note that (weighed, with $w_i\geq 0$) $L_p$ norms inducing
metrics
\begin{equation}
    d_{\ell_p} (a,b) = \left ( \sum_{i=0}^n w_i|a_i-b_i |^p \right )^{\frac{1}{p}}\,
  \end{equation}
characterize metric spaces $(\Z^n, \R , d_{\ell_p})$. However,  note that
$(\Z^n,\Z, \lfloor d_{\ell_p}) \rfloor)$ is a metric space.


\section{Separable Distance Transformation}
\label{sec:separ-dist-transf}

\subsection{A First Generic Adapter}
\label{sec:first-gener-adapt}

In

\subsection{Subquadratic Algorithm in Dimension 2}
\label{sec:subq-algor-dimens}


\subsection{Subquadratic Algorithms in Higher Dimension}
\label{sec:subq-algor-high}


\section{Experimental Analysis}
\label{sec:exper-analys}

\subsection{Implementation details}
\label{sec:impl-deta}


\subsection{Voronoi Maps in dimension 2}
\label{sec:impl-2D}



\section{Discussion}
\label{sec:discussion}



\bibliographystyle{splncs}
\bibliography{library,mybiblio,dtbib}
\end{document}
